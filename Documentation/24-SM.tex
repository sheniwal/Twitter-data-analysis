\documentclass[]{article}

%opening
\title{Sentiment Analysis of Tweets and News Media Reporting on Demonetization Issues}

\author{Ambuj Mishra,Sheshan Sheniwal,Sunil Kumar}




\begin{document}

\maketitle
\begin{center}
{\large\bf
Group No   -  24\\
\vspace{2ex}  
Supervisor Name   -  Dr. Pradip Swarnakar}  \\



\end{center}


\section{Problem Statement}
Compare the sentiments of online tweets and News reports on the topic “Demonetization” and find the relationship between them on the basis of their sentiments.
\section{Methodology}
Firstly, we will scrap tweets on “Demonetization” which were tweeted between 9th November to 31st December. Then, we will extract and separate the data based on date, id, screen-name using different Python libraries. We will refine the data using Natural Language Toolkit (NLTK) and Python libraries. Then, we will apply different algorithms for analyzing the sentiments of the tweets in terms of positive, negative and neutral outputs. Using MATLAB, we will draw graphs based on the results.
\newline
Now, we will collect data from news reports which were published during 9th November to 31st December on the topic “Demonetization” specifically. Then, we will use the same techniques as for the tweets to analyze the sentiments of news reports in terms of positive, negative and neutral outputs. Again, we will draw graphs based on the results.
\newline
Lastly, we will compare the sentiments of tweets and news reports and will find the relationship between them. And then, we will draw comparative graphs based on the results.
\section{Experimental Setup (Dataset, Software, Hardware Used)}
{\bf Dataset} : Dataset of online tweets and different news articles on "Demonetization" (The training data contains data between 9th November to 31st Decemeber 2016).
\newline
{\bf Software used} : pycharm ide, Canopy, Sublime Text 3, NLTK, Pyth, numpy, scipy,stopwords, Windows 10, ubuntu 16.06
\end{document}
